%% Generated by Sphinx.
\def\sphinxdocclass{report}
\documentclass[letterpaper,10pt,italian]{sphinxmanual}
\ifdefined\pdfpxdimen
   \let\sphinxpxdimen\pdfpxdimen\else\newdimen\sphinxpxdimen
\fi \sphinxpxdimen=.75bp\relax

\PassOptionsToPackage{warn}{textcomp}
\usepackage[utf8]{inputenc}
\ifdefined\DeclareUnicodeCharacter
 \ifdefined\DeclareUnicodeCharacterAsOptional
  \DeclareUnicodeCharacter{"00A0}{\nobreakspace}
  \DeclareUnicodeCharacter{"2500}{\sphinxunichar{2500}}
  \DeclareUnicodeCharacter{"2502}{\sphinxunichar{2502}}
  \DeclareUnicodeCharacter{"2514}{\sphinxunichar{2514}}
  \DeclareUnicodeCharacter{"251C}{\sphinxunichar{251C}}
  \DeclareUnicodeCharacter{"2572}{\textbackslash}
 \else
  \DeclareUnicodeCharacter{00A0}{\nobreakspace}
  \DeclareUnicodeCharacter{2500}{\sphinxunichar{2500}}
  \DeclareUnicodeCharacter{2502}{\sphinxunichar{2502}}
  \DeclareUnicodeCharacter{2514}{\sphinxunichar{2514}}
  \DeclareUnicodeCharacter{251C}{\sphinxunichar{251C}}
  \DeclareUnicodeCharacter{2572}{\textbackslash}
 \fi
\fi
\usepackage{cmap}
\usepackage[T1]{fontenc}
\usepackage{amsmath,amssymb,amstext}
\usepackage{babel}
\usepackage{times}
\usepackage[Sonny]{fncychap}
\ChNameVar{\Large\normalfont\sffamily}
\ChTitleVar{\Large\normalfont\sffamily}
\usepackage{sphinx}

\usepackage{geometry}

% Include hyperref last.
\usepackage{hyperref}
% Fix anchor placement for figures with captions.
\usepackage{hypcap}% it must be loaded after hyperref.
% Set up styles of URL: it should be placed after hyperref.
\urlstyle{same}

\addto\captionsitalian{\renewcommand{\figurename}{Fig.}}
\addto\captionsitalian{\renewcommand{\tablename}{Tabella}}
\addto\captionsitalian{\renewcommand{\literalblockname}{Listato}}

\addto\captionsitalian{\renewcommand{\literalblockcontinuedname}{continua dalla pagina precedente}}
\addto\captionsitalian{\renewcommand{\literalblockcontinuesname}{continues on next page}}

\addto\extrasitalian{\def\pageautorefname{pagina}}

\setcounter{tocdepth}{1}



\title{GovPay Documentation}
\date{14 mag 2019}
\release{3.1.0}
\author{Link.it}
\newcommand{\sphinxlogo}{\vbox{}}
\renewcommand{\releasename}{Release}
\makeindex

\begin{document}
\ifdefined\shorthandoff
  \ifnum\catcode`\=\string=\active\shorthandoff{=}\fi
  \ifnum\catcode`\"=\active\shorthandoff{"}\fi
\fi
\maketitle
\sphinxtableofcontents
\phantomsection\label{\detokenize{index::doc}}



\chapter{Introduzione}
\label{\detokenize{introduzione/index:introduzione}}\label{\detokenize{introduzione/index:inst-intro}}\label{\detokenize{introduzione/index::doc}}
Questo manuale fornisce le informazioni generali e la procedura
necessaria per l’installazione e il dispiegamento di GovPay. Tale
procedura prevede una fase preliminare di verifica dei requisiti di
installazione sull’ambiente di destinazione, seguita da una fase di
configurazione dei moduli applicativi tramite un installer grafico, per
poi concludere con la fase di deploy nell’ambiente di destinazione.

Terminata la procedura di installazione vengono descritti i passi da
effettuare per verificarne la riuscita.


\chapter{Ambiente e architettura di riferimento}
\label{\detokenize{ambiente/index:ambiente-e-architettura-di-riferimento}}\label{\detokenize{ambiente/index:inst-ambiente}}\label{\detokenize{ambiente/index::doc}}
L’ambiente di esecuzione di GovPay richiede la presenza di software di
base, precedentemente installato i cui riferimenti sono:
\begin{itemize}
\item {} 
JVM Oracle Java 8

\item {} 
Application Server WildFly 11

\end{itemize}

Relativamente alla versione di \sphinxstylestrong{RDBMS}, l’attuale versione di
GovPay consente di selezionare tra i seguenti:
\begin{itemize}
\item {} 
\sphinxstylestrong{PostgreSQL 8.x o superiore}

\item {} 
\sphinxstylestrong{MySQL 5.6.4 o superiore}

\item {} 
\sphinxstylestrong{Oracle 10g o superiore}

\end{itemize}


\chapter{Configurazione dei moduli applicativi}
\label{\detokenize{configurazione/index:configurazione-dei-moduli-applicativi}}\label{\detokenize{configurazione/index:inst-configurazione}}\label{\detokenize{configurazione/index::doc}}
La fase di configurazione dei moduli applicativi consente di impostare i
dati di riferimento del proprio ambiente di installazione, tramite una
procedura basata sul modello wizard.


\section{Download}
\label{\detokenize{configurazione/index:download}}
Scaricare l’ultima versione (binary release) di GovPay dal sito GitHub
\sphinxurl{https://github.com/link-it/GovPay}.


\section{Esecuzione dell’Installer}
\label{\detokenize{configurazione/index:esecuzione-dell-installer}}
L’archivio di installazione può essere scompattato e il relativo
installer eseguito su un ambiente che non deve essere necessariamente
quello di destinazione. Infatti l’Installer non installa il prodotto ma
produce tutti gli elementi necessari che dovranno essere dispiegati
nell’ambiente di esercizio.

Per l’esecuzione dell’installer verificare ed eventualmente impostare la
variabile d’ambiente \sphinxstylestrong{JAVA\_HOME} in modo che riferisca la directory
radice dell’installazione di Java. Eseguire quindi l’installer mandando
in esecuzione il file \sphinxstylestrong{install.sh} su Unix/Linux, oppure
\sphinxstylestrong{install.cmd} su Windows.


\subsection{Avvio}
\label{\detokenize{configurazione/index:avvio}}
L’Installer mostra all’avvio una pagina introduttiva.

Sono mostrate informazioni quali:
\begin{itemize}
\item {} 
Nome e versione del prodotto

\item {} 
Informazioni sul copyright

\item {} 
Informazioni sulla licenza d’uso

\end{itemize}

Selezionando il pulsante Next si procede con la configurazione del
software.

\begin{figure}[htbp]
\centering
\capstart

\noindent\sphinxincludegraphics[width=14.224cm,height=10.211cm]{{configurazione/../_figure_installazione/100002010000022600000192C7342CEDBB4934E5}.png}
\caption{Figura 1: Pagina introduttiva all’avvio dell’Installer}\label{\detokenize{configurazione/index:id1}}\end{figure}


\subsection{Informazioni Preliminari}
\label{\detokenize{configurazione/index:informazioni-preliminari}}
La schermata «Informazioni Preliminari» consente di inserire i dati sul
contesto di installazione nell’ambiente di esercizio.

\begin{figure}[htbp]
\centering
\capstart

\noindent\sphinxincludegraphics[width=14.224cm,height=10.363cm]{{configurazione/../_figure_installazione/100002010000022700000192CD0548360449197A}.png}
\caption{Figura 2: Informazioni Preliminari}\label{\detokenize{configurazione/index:id2}}\end{figure}

Devono essere inserite le seguenti informazioni:
\begin{itemize}
\item {} 
\sphinxstylestrong{Application Server:** **la scelta dell’application server è
vincolata su «WildFly 11.0»}

\item {} 
\sphinxstylestrong{Work** Folder: **inserire il path assoluto della
**directory**, presente nell’ambiente di
**destinazione**,** **che sarà **utilizzata da GovPay
**per accedere a** dati accessori **legati alle**
funzionalità opzionali, ad esempio:}
\begin{itemize}
\item {} 
\sphinxstylestrong{file di configurazione personalizzati}

\item {} 
\sphinxstylestrong{loghi dei psp}

\end{itemize}

\item {} 
\sphinxstylestrong{Log Folder}: inserire il path assoluto della directory, presente
nell’ambiente di destinazione, che sarà utilizzata da GovPay per
inserire i diversi file di tracciamento prodotti.

\end{itemize}


\subsection{Informazioni Applicative}
\label{\detokenize{configurazione/index:informazioni-applicative}}\begin{itemize}
\item {} 
\sphinxstyleemphasis{Username Amministratore:} indicare l’identificativo dell’utenza di
amministrazione per l’accesso alla console di gestione e
monitoraggio. Tipicamente si fornisce il «principal» dell’utenza
applicativa registrata sull’Application Server, ma è in alternativa
possibile indicare altre tipologie di utenze, come ad esempio
identificate dal Certificato Client Digitale (maggiori dettagli in
merito vengono forniti più avanti).

\item {} 
\sphinxstyleemphasis{Nome Dominio:} inserire l’hostname tramite il quale saranno
raggiungibili i servizi di GovPay (ad esempio la console di
monitoraggio).

\end{itemize}

\begin{figure}[htbp]
\centering
\capstart

\noindent\sphinxincludegraphics[width=14.143cm,height=9.959cm]{{configurazione/../_figure_installazione/100002010000022700000192D4FF505CBCE8C644}.png}
\caption{Figura 3: Informazioni Applicative}\label{\detokenize{configurazione/index:id3}}\end{figure}


\subsection{Il Database}
\label{\detokenize{configurazione/index:il-database}}
Nella schermata «Il Database» si devono inserire i riferimenti per
l’accesso al database di esercizio di GovPay.

\begin{figure}[htbp]
\centering
\capstart

\noindent\sphinxincludegraphics[width=14.21cm,height=10.183cm]{{configurazione/../_figure_installazione/100002010000022600000192A2989B695B3A28EB}.png}
\caption{Figura 4: Informazioni Accesso Database}\label{\detokenize{configurazione/index:id4}}\end{figure}
\begin{itemize}
\item {} 
\sphinxstylestrong{DB Platform:} selezionare la piattaforma RDBMS utilizzata

\item {} 
\sphinxstylestrong{Hostname}: indirizzo per raggiungere il database

\item {} 
\sphinxstylestrong{Porta}: la porta da associare all’hostname per la connessione al
database

\item {} 
\sphinxstylestrong{Nome Database}: il nome dell’istanza del database a supporto di
GovPay.

\item {} 
\sphinxstylestrong{Username}: l’utente con diritti di lettura/scrittura sul database
sopra indicato.

\item {} 
\sphinxstylestrong{Password}: la password dell’utente del database.

\end{itemize}

\begin{sphinxadmonition}{note}{Nota:}
Non è necessario che il database e l’utente indicato esistano in questa fase. Potranno essere creati nella successiva fase di dispiegamento purché i dati relativi coincidano con i valori inseriti in questi campi del wizard.
\end{sphinxadmonition}


\subsection{Installazione}
\label{\detokenize{configurazione/index:installazione}}
Premendo il pulsante \sphinxstylestrong{Install} il processo di configurazione termina
con la produzione dei files necessari per l’installazione di GovPay che
verranno inseriti nella nuova directory \sphinxstylestrong{dist} creata al termine di
questo processo.

\begin{figure}[htbp]
\centering
\capstart

\noindent\sphinxincludegraphics[width=14.446cm,height=10.53cm]{{configurazione/../_figure_installazione/1000020100000227000001912C8859F6CB3B2892}.png}
\caption{Figura 5: Installazione Terminata}\label{\detokenize{configurazione/index:id5}}\end{figure}

I files presenti nella directory \sphinxstylestrong{dist} dovranno essere utilizzati
nella fase successiva di dispiegamento di GovPay.


\chapter{Fase di Dispiegamento}
\label{\detokenize{dispiegamento/index:fase-di-dispiegamento}}\label{\detokenize{dispiegamento/index:inst-dispiegamento}}\label{\detokenize{dispiegamento/index::doc}}
Al termine dell’esecuzione dell’utility di installazione vengono
prodotti i files necessari per effettuare il dispiegamento nell’ambiente
di esercizio. Tali files sono disponibili nella directory \sphinxstylestrong{dist}
prodotta dall’utility.

Per il dispiegamento nell’ambiente di destinazione devono essere
effettuati i seguenti passi:
\begin{enumerate}
\item {} 
Creare un utente sul RDBMS avente i medesimi valori di username e
password indicati in fase di setup.

\item {} 
Creare un database, per ospitare le tabelle dell’applicazione, avente
il nome indicato durante la fase di setup. Il charset da utilizzare è
UTF-8.

\item {} 
Impostare i permessi di accesso in modo che l’utente creato al passo
1 abbia i diritti di lettura/scrittura sul database creato al \sphinxstylestrong{passo
2}.

\item {} 
Garantire la raggiungibilità dell’application server al RDBMS
indicato in fase di setup.

\item {} 
Eseguire lo script \sphinxstylestrong{sql/gov\_pay.sql} per la creazione dello schema
del database. Ad esempio, nel caso di PostgreSQL, si potrà eseguire
il comando:
\begin{itemize}
\item {} 
\sphinxstylestrong{psql -h \textless{}hostname\textgreater{} -d \textless{}database\textgreater{} -U \textless{}username\textgreater{} -f sql/gov\_pay.sql}

\end{itemize}

\end{enumerate}
\begin{enumerate}
\item {} 
In riferimento al valore indicato come «Username
Amministratore», creare l’utenza
applicativa sull’application server che
rappresenti l’amministratore di GovPay. Per farlo è possibile
utilizzare lo script presente nella distribuzione di WildFly
in ./bin/add-user.sh o ./bin/add-user.bat, fornendo i
seguenti parametri:
\begin{itemize}
\item {} 
\sphinxstyleemphasis{Type of user}: indicare b) Application User

\item {} 
\sphinxstyleemphasis{Realm}: lasciare il valore di default

\item {} 
\sphinxstyleemphasis{Username}: utenza amministratore di GovPay indicata durante
l’esecuzione dell’Installer (es. Gpadmin)

\item {} 
\sphinxstyleemphasis{Password}: password associata all’utenza

\item {} 
\sphinxstyleemphasis{Roles}: lasciare il valore di default

\item {} 
\sphinxstyleemphasis{Group:} lasciare il valore di default

\item {} 
\sphinxstyleemphasis{Is this new user going to be used for one AS process to connect
to another AS process?: Indicare “no”}.

\end{itemize}

\end{enumerate}
\begin{enumerate}
\item {} 
Copiare il file \sphinxstylestrong{datasource/govpay-ds.xml}, contenente la
definizione del datasource, nella directory
\sphinxstylestrong{\textless{}JBOSS\_HOME\textgreater{}/standalone/deployments} dell’application server.

\item {} 
Copiare le applicazioni presenti nella directory \sphinxstylestrong{archivi} nella
directory \sphinxstylestrong{\textless{}JBOSS\_HOME\textgreater{}/standalone/deployments} dell’application server.

\item {} 
Installare il DriverJDBC, relativo al tipo di RDBMS indicato in fase
di setup, nella directory \sphinxstylestrong{\textless{}JBOSS\_HOME\textgreater{}/standalone/deployments} dell’application server.

\item {} 
Editare i datasources installati al \sphinxstylestrong{punto 7}. sostituendo la
keyword \sphinxstylestrong{NOME\_DRIVER\_JDBC.jar} con il nome del file corrispondente
al driver jdbc.

\item {} 
Verificare che la directory di lavoro e quella di log di GovPay,
inserite in fase di configurazione, esistano o altrimenti crearle con
permessi tali da consentire la scrittura all’utente di esecuzione del
processo java dell’application server.

\item {} 
Avviare l’application server (ad esempio su Linux con il comando
\sphinxstylestrong{\textless{}JBOSS\_HOME\textgreater{}/bin/standalone.sh} oppure utilizzando il relativo
service).

\end{enumerate}


\chapter{Verifica dell’Installazione}
\label{\detokenize{verifica/index:verifica-dell-installazione}}\label{\detokenize{verifica/index:inst-verifica}}\label{\detokenize{verifica/index::doc}}
Per la fase di verifica dell’installazione, effettuare i seguenti passi:
\begin{enumerate}
\item {} 
Avviare l’application server

\item {} 
Al termine della fase di avvio, sono riscontrabili i seguenti
contesti dispiegati, suddivisi tra servizi di frontend (rivolti
all’utente finale) e servizi di backend (rivolti all’utenza interna):

\end{enumerate}
\begin{itemize}
\item {} \begin{itemize}
\item {} 
\sphinxstylestrong{Frontend:}
\begin{itemize}
\item {} 
\sphinxstylestrong{/govpay/frontend/web/connector}

\sphinxstylestrong{web application per la gestione delle redirezioni durante i
flussi di pagamento}

\item {} 
\sphinxstylestrong{/govpay/frontend/api/pagamento}

\sphinxstylestrong{api per l’esecuzione dei pagamenti da parte del debitore}

\item {} 
\sphinxstylestrong{/govpay/frontend/api/pagopa}

\sphinxstylestrong{api per la gestione del colloquio con la piattaforma centrale
pagoPA}

\end{itemize}

\item {} 
\sphinxstylestrong{Backend:}
\begin{itemize}
\item {} 
\sphinxstylestrong{/govpay/backend/api/pendenze}

\sphinxstylestrong{api per la gestione dell’archivio dei pagamenti in attesa
(pendenze, pagamenti, ecc.)}

\item {} 
\sphinxstylestrong{/govpay/backend/api/ragioneria}

\sphinxstylestrong{api relative ai servizi di riconciliazione degli incassi con
le pendenze/pagamenti di origine}

\item {} 
\sphinxstylestrong{/govpay/backend/api/backoffice}

\sphinxstylestrong{api relative ai servizi di configurazione della piattaforma
(domini, applicazioni, operatori, ecc.)}

\item {} 
\sphinxstylestrong{/govpay/backend/gui/backoffice}

\sphinxstylestrong{web application che corrisponde al cruscotto di gestione e
monitoraggio di GovPay}

\end{itemize}

\end{itemize}

\end{itemize}
\begin{enumerate}
\item {} 
Verificare che i servizi esposti da GovPay verso pagoPA siano
raggiungibili verificando sul browser le seguenti URL:

\end{enumerate}
\begin{itemize}
\item {} 
\sphinxurl{http:/}/\textless{}hostname\textgreater{}:\textless{}port\textgreater{}/govpay/frontend/api/pagopa/PagamentiTelematiciCCPservice?wsdl

\item {} 
\sphinxurl{http:/}/\textless{}hostname\textgreater{}:\textless{}port\textgreater{}/govpay/frontend/api/pagopa/PagamentiTelematiciRTservice?wsdl

\end{itemize}
\begin{enumerate}
\item {} 
Verificare che la \sphinxstylestrong{govpayConsole}, l’applicazione web per la
gestione della configurazione e monitoraggio di GovPay, sia
accessibile tramite browser all’indirizzo:

\end{enumerate}
\begin{itemize}
\item {} 
\sphinxstylestrong{http://\textless{}hostname\textgreater{}:\textless{}port\textgreater{}/govpay/backend/gui/backoffice}

\end{itemize}

In caso di corretto funzionamento verrà visualizzata la pagina di
autenticazione per l’accesso alla console.
\begin{enumerate}
\item {} 
Accedere alla govpayConsole usando l’utenza di jboss configurata in
fase di dispiegamento.

L’utenza creata in precedenza ha accesso a tutte le funzionalità
compresa la gestione degli utenti. Utilizzando questo accesso
potranno quindi essere registrati dei nuovi utenti.

\item {} 
Completata l’installazione di GovPay, per proseguire con l’utilizzo
del sistema si rimanda al “Manuale Utente”.

\end{enumerate}


\chapter{Configurazione in Load Balancing}
\label{\detokenize{loadbalancing/index:configurazione-in-load-balancing}}\label{\detokenize{loadbalancing/index:inst-loadbalancing}}\label{\detokenize{loadbalancing/index::doc}}
Per realizzare un’installazione in load balancing è necessario
predisporre più istanze dell’Application Server, ognuna con una propria
installazione di GovPay. Sarà inoltre necessario:
\begin{enumerate}
\item {} 
Che tutte le istanze di GovPay siano configurate per condividere lo stesso DB.

\item {} 
Che esista un Load Balancer in grado di bilanciare il flusso di richieste in arrivo sulle varie istanze di application server ospitanti GovPay.

\item {} 
Che GovPay sia opportunamente configurato con un identificatore unico che contraddistingua lo specifico nodo.

\end{enumerate}

Le proprietà per la configurazione del singolo nodo sono le seguenti:
\begin{itemize}
\item {} 
\sphinxstylestrong{it.govpay.clusterId}: identificativo dell’istanza di GovPay. Deve essere un numero univoco tra le istanze.

\item {} 
\sphinxstylestrong{it.govpay.timeoutBatch}: timeout in secondi delle operazioni soggette alla gestione applicativa della concorrenza. Se non valorizzato viene usato il default di 5 minuti.

\end{itemize}

Queste proprietà possono essere specificate sia nelle Java Options, dei processi Java associati agli application server, oppure nel file \sphinxstyleemphasis{govpay.properties} nella directory di lavoro di ogni nodo.

Ad esempio è possibile ridefinire la directory di log impostando la seguente proprietà:
\begin{itemize}
\item {} 
it.govpay.resource.log.path

\end{itemize}


\chapter{Servizi di Monitoraggio}
\label{\detokenize{monitoraggio/index:servizi-di-monitoraggio}}\label{\detokenize{monitoraggio/index:inst-monitoraggio}}\label{\detokenize{monitoraggio/index::doc}}
Per consentire l’integrazione con i sistemi di monitoraggio, GovPay
mette a disposizione servizi interrogabili per verificare il
funzionamento del sistema.

I servizi di monitoraggio sono di due tipi:
\begin{itemize}
\item {} 
Monitoraggio Domini

per verificare l’esito delle ultime comunicazioni con il Nodo dei
Pagamenti, relativamente ad uno specifico dominio.

\item {} 
Monitoraggio GovPay

per verificare il funzionamento delle singole componenti del
prodotto.

\end{itemize}


\section{Monitoraggio domini}
\label{\detokenize{monitoraggio/index:monitoraggio-domini}}
Viene esposto un servizio di monitoraggio per dominio che fornisce
indicazioni di stato inerenti l’esito delle interazioni con il Nodo dei
Pagamenti. Il servizio si interroga con la seguente chiamata HTTP:

GET /govpay/frontend/api/pagopa/rs/check/\{id\_dominio\} HTTP/1.1

Accept: application/json

in ritorno si ha un messaggio con questo formato:

\{

«ultimo\_aggiornamento»:null,

«codice\_stato»:1,

«operazione\_eseguita»:null,

«errore\_rilevato»:»STATO NON VERIFICATO»

\}

con la seguente semantica:


\begin{savenotes}\sphinxattablestart
\centering
\begin{tabulary}{\linewidth}[t]{|T|T|}
\hline

ultimo\_aggiornamento
&
Data dell’ultimo aggiornamento
dello stato
\\
\hline
codice\_stato
&
0: ok

1: stato non verificato

2: fail
\\
\hline
operazione\_eseguita
&
Operazione richiesta al nodo che
ha aggiornato lo stato
\\
\hline
errore\_rilevato
&
Dettaglio dell’errore riscontrato
\\
\hline
\end{tabulary}
\par
\sphinxattableend\end{savenotes}


\section{Monitoraggio GovPay}
\label{\detokenize{monitoraggio/index:monitoraggio-govpay}}
Sono implementati dei check sui servizi gestiti da GovPay per
verificarne il corretto funzionamento. Lo stato dei check è consultabile
tramite servizi REST.
\begin{quote}

GET /govpay/frontend/api/pagopa/rs/check/sonda/
\end{quote}

Il servizio restituisce una panoramica dei check attivi sul sistema e
del loro stato attuale. Per ciascuno è possibile acquisirne il
dettaglio:

GET /govpay/frontend/api/pagopa/rs/check/sonda/\{nome\}

dove \sphinxstyleemphasis{nome} può assumere i seguenti valori:


\begin{savenotes}\sphinxattablestart
\centering
\begin{tabulary}{\linewidth}[t]{|T|T|}
\hline

update-psp
&
Check del servizio di aggiornamento PSP
\\
\hline
update-rnd
&
Check del servizio di acquisizione flussi rendicontazione
\\
\hline
update-pnd
&
Check del servizio di risoluzione pagamenti pendenti
\\
\hline
update-ntfy
&
Check del servizio di spedizione notifiche
\\
\hline
update-conto
&
Check del servizio di generazione estratti conto
\\
\hline
check-ntfy
&
Check della coda di notifiche da spedire
\\
\hline
\end{tabulary}
\par
\sphinxattableend\end{savenotes}

in ritorno si ha un messaggio con questo formato:

\{
\begin{quote}

«nome»:»check-ntfy»,

«stato»:0,

«descrizioneStato»:null,

«durataStato»:null,

«sogliaWarn»:»Numero di elementi accodati: 10»,

«sogliaError»:»Numero di elementi accodati: 100»,

«sogliaWarnValue»:10,

«sogliaErrorValue»:100,

«dataUltimoCheck»:1489673880116,

«tipo»:»Coda»
\end{quote}

\}

con la seguente semantica:


\begin{savenotes}\sphinxattablestart
\centering
\begin{tabulary}{\linewidth}[t]{|T|T|}
\hline

Nome
&
Identificativo della check
\\
\hline
stato
&
null: stato non verificato

0: ok

1: warning

2: error
\\
\hline
descrizioneStato
&
Descrizione informativa sullo
stato assunto dal check
\\
\hline
durataStato
&
Tempo in millisecondi in cui il
check e” nello stato attuale
\\
\hline
sogliaWarn
&
Soglia di Warning in forma
descrittiva
\\
\hline
sogliaError
&
Soglia di Error in forma
descrittiva
\\
\hline
sogliaWarnValue
&
Valore di soglia per lo stato di
warning. La semantica del valore
dipende dal tipo di check:
\\
\hline
sogliaError
&
Come \sphinxstyleemphasis{sogliaWarnValue} ma per lo
stato di error
\\
\hline
dataUltimoCheck
&
Data dell’ultima verifica del
check
\\
\hline
tipo
&
Tipologia di check:
\\
\hline
\end{tabulary}
\par
\sphinxattableend\end{savenotes}



\renewcommand{\indexname}{Indice}
\printindex
\end{document}